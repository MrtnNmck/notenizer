\newpage
\thispagestyle{plain}
\begin{center}
\begin{Large}
\textbf{Anotácia} \\
\end{Large}
\end{center}
Slovenská technická univerzita v Bratislave \\
FAKULTA INFORMATIKY A INFORMAČNÝCH TECHNOLÓGIÍ \\
\noindent
Študijný program: \Program \\
\noindent
Autor: \Author \\
\ifthenelse {\boolean{bachelor}}
{
	{Bakalárska práca: }\Title \\
}
{
	{Diplomová práca: }\Title \\
}
Vedúci práce: \Supervisor \\
\Month \Year \\
\noindent
\\
Sme zavalení množstvom informácií z rôznych zdrojov. V procese výučby je náročné vytvoriť poznámky, ktoré zahŕňajú podmnožinu dôležitých informácií zo zdroja. Existuje niekoľko spôsobov, ako extrahovať informácie z textu. V tejto práci navrhujeme systém na extrakciu informácií a tvorbu poznámok z textu, ktoré sú dôležité z pohľadu výučby. Systém umožní študentom vytvárať personalizované poznámky. Využívame najmä syntaktickú analýzu textu a poznámky sa vytvárajú pomocou využitia značiek slovných druhov, názvoslovných entít, lema tvarov slov a závislostí medzi slovami vo vetách. Výsledkom je interaktívny systém na tvorbu poznámok, ktoré sa vytvárajú na základe pravidiel naučených od používateľa. Pravidlo sa skladá najmä zo závislostí, je závislé od štruktúry viet a aplikovateľné na viacero viet. Overili sme eliminovanie irelevantných informácií pomocou pravidiel, aplikovanie pravidiel na viacero viet a zlepšovanie systému so zvyšujúcim sa počtom pravidiel z hľadiska spoľahlivosti a personalizácie.
\newpage
\thispagestyle{plain}
\begin{center}
\begin{Large}
\textbf{Annotation} \\
\end{Large}
\end{center}
Slovak University of Technology Bratislava \\
FACULTY OF INFORMATICS AND INFORMATION TECHNOLOGIES \\
\noindent
Degree Course: Informatics \\
\noindent
Author: \Author \\
\ifthenelse {\boolean{bachelor}}
{
	{Bachelor thesis: Educational texts processing} \\
}
{
	{Master thesis: }\Title \\
}
Supervisor: \Supervisor \\
May \Year \\
\noindent
\\
We are overwhelmed by information from various topics. The challenge in education is to create notes which cover important subset of information. There are known methods to extract information from text. In this thesis we propose a system to extract the notes from text which are important for educational purpose. System allows students to create personalized notes. We use mainly syntactic text analysis and the notes are created by help of part-of-speech tags, named entities, lemmas and dependencies between words in sentences. The outcome is an interactive system for creating notes based on the learned rules from the user. The rule consists mainly from dependencies, depends on the structure of sentence and is applicable on various sentences. We verified the eleimination of the irelevant information by the rules, aplication of the rules on various sentences and enhancements to reliability and personalization by the increased amount of the rules.