\newpage
\ifthenelse {\boolean{bachelor}}
{
	%\section{Analysis}
	\section{Analýza} 
}
{
	%\chapter{Analysis}
	\chapter{Analyza}
}
V tejto kapitole priblížím a rozoberiem čo je spracovanie prirodzeného jazyka, kde a ako sa dá využiť. Ďalej zanalyzujem existujúce aplikácie a systému, ktorých základom je spracovanie textu.
\ifthenelse {\boolean{bachelor}}
{
	%\subsection{Subsection}
	\subsection{Spracovanie prirodzeného jazyka}
}
{
	%\section{Subsection}
	\section{Spracovanie prirodzeného jazyka}
}
\label{subsec:nlp}
Spracovanie prirodzeného jazyka (angl. Natural Language Processing - NLP) je oblasť vedného oboru, ktorý sa zaoberá počítačovými technológiami. Hlavným cieľom NLP je dosiahnuť aby počítač dokázal porozumieť ľudskej, či už písanej alebo hovorovej, reči.

Porozumenie ľudskej reči je mnohokrát náročne aj pre samotných ľudí a nie to ešte pre počítače. Na svete je veľké množstvo jazykov, ktoré sa od seba líšia charakteristikami typickými pre konkrétny jazyk. Taktiež každý človek sa líši a preto výslovnosť rovnakého slova viacerými ľuďmi môže byť odlišná. Ďalej máme slangové slová a slová typické len pre určité územie. Pri spracovávaní prirodzeného jazyka treba vziať do úvahy tieto a aj ďalšie premenné. Dosiahnutie tohto cieľa je preto často veľmi náročné.

V súčastnosti najpoužívanejšie algoritmy na NLP využívajú strojové učenie. Dosiahnutie úplného porozumenia a spracovania ľudského prirodzeného jazyka by znamenalo vyriešiť \textit{AI-complete} problém, čo znamená, že obtiažnosť tohto problému je ekvivalentné z obtiažnosti problému vytvorenia počítača inteligentného ako človek, takzvané ,,true AI''.

NLP ma niekoľko hlavných úloh. Podrobnejšie si priblížime tie, ktoré sú relevantné vzhľadom na implementáciu spracovania učebných textov.

Úlohy spracovania prirodzeného textu: [Natural Language Processing (Almost) from Scratch, Ronan Collobert]
\begin{itemize}
	\item Značkovanie slovných druhov (angl. Part-of-speech tagging) \ref{subsubsec:postagging}
	\item Rozdelenie vety na menšie časti (angl. Chunking)
	\item Rozpoznávanie názvoslovných entít (angl. Named Entity Recognition) \ref{subsubsec:ner}
	\item Označovanie sémantického postavenie (angl. Semantic Role Labeling)
	\item Rozpoznanie koreferencií (angl. Coreference resolution) \ref{subsubsec:corefparsing}
	\item Morfologické segmentovanie (angl. Morphological Segmentation)
	\item Generovanie prirodzeného jazyka (angl. Natural Language Generation)
	\item Optické rozoznávanie textu (angl. Optical Character Recognition)
	\item Rozloženie vzťahov (angl. Dependency parsing) \ref{subsubsec:corefparsing}
	\item a mnoho ďalších
\end{itemize}

\ifthenelse {\boolean{bachelor}}
{
	%\subsection{Subsection}
	\subsubsection{Značkovanie slovných druhov}
}
{
	%\section{Subsection}
	\subsection{Značkovanie slovných druhov}
}
\label{subsubsec:postagging}
Hlavnou úlohou značkovania slovných druhom (angl. Part-of-speech tagging) je každému slovu vo vete priradiť unikátnu značku, ktorá odrážať jeho syntaktickú úlohu vo vete. Sú to, napríklad v slovenskom jazyku podmet, prísudok, príslovkové určenie alebo v anglickom jazyku noun, adverb, verb, atď. Tak isto to môže byť označenie určujúce množné číslo, napríklad signulár alebo plurál.

Problémom pri značkovaní slovných druhov je mnohoznačnosť. Znamená to, že slovo môže mať viacero významov a môže byť viacerými slovnými druhmi. Napríklad v slovenskom jazyku slovo \textit{kry} môže predstavovať sloveso s významom rozkazu \textit{prikry!}, ale taktiež môže predstavovať podstatné meno s významom \textit{kríky}. V anglickom jazyku to je napríklad slovo \textit{book}, ktoré môže predstavovať podstatné meno (angl. noun) \textit{kniha} alebo sloveso (angl. verb) vo význame \textit{rezervovať}.

\ifthenelse {\boolean{bachelor}}
{
	%\subsection{Subsection}
	\subsubsection{Rozpoznávanie názvoslovných entít}
}
{
	%\section{Subsection}
	\subsection{Rozpoznávanie názvoslovných entít}
}
\label{subsubsec:ner}
Rozpoznávanie názvoslovných entít (angl. Named Entity Recognition) označuje mená a názvy, ktoré sa vyskytujú v texte. Rozdeľuje tieto entity do kategórií, ako sú napríklad \textit{osoby}, \textit{organizácie} alebo \textit{lokácie}.

Ťažkosti pri rozpoznávaní názvoslovných entít spôsobuje kapitalizácia slov, takzvané písanie entít s veľkým začiatočným písmenom. V anglickom jazyku to jednoduché, kedže v angličtine sa entity píšu s veľkým začiatočným písmenom. Príklad je \textit{Slovak University of Technology}. Avšak v iných jazykoch to neplatí a entity sa nemusia písať s veľkým začiatočným písmenom.

\ifthenelse {\boolean{bachelor}}
{
	%\subsection{Subsection}
	\subsubsection{Rozpoznanie koreferencií}
}
{
	%\section{Subsection}
	\subsection{Rozpoznanie koreferencií}
}
\label{subsubsec:corefparsing}
Nájdenie, identifikácia a rozpoznanie koreferencií v texte je úlohou rozpoznávania koreferencií (angl. Coreference resolution). V texte sa často používajú zámena (angl. pronouns) \textit{to}, \textit{tí}, \textit{on} alebo anglicky \textit{it}, \textit{those}, \textit{he} a mnoho ďalších. Tieto zámena sa odkazujú na iné podstatné mená alebo mená a názvy a je úlohou rozpoznávania koreferencií určiť, na ktoré podstatné meno alebo meno, alebo názov sa konkrétne zámeno odkazuje.

Príklad:
\textbf{Martin Nemček} napísal túto bakalársku prácu. \textbf{On} študuje na FIIT STU BA.

Tu je vidno, že zámeno \textit{on} sa odkazuje na meno \textit{Martin Nemček}.

\ifthenelse {\boolean{bachelor}}
{
	%\subsection{Subsection}
	\subsubsection{Rozloženie vzťahov}
}
{
	%\section{Subsection}
	\subsection{Rozloženie vzťahov}
}
\label{subsubsec:dependencyparsing}
Rozloženie na vzťahy nám poskytuje jednoduchý opis gramatických vzťahov slov vo vete. Aplikovaním rozloženia vzťahov na vetu \textit{Bell, based in Los
Angeles, makes and distributes electronic, computer and building products.
} vznikne strom vzťahov (angl. dependency tree) (viď. \ref{fig:dependency_tree}).

\begin{figure}[H]
\begin{center}\includegraphics[scale=0.8]{dependency_tree}\end{center}
\caption[Strom vzťahov]{Strom vzťahov}\label{fig:dependency_tree}
\end{figure}

V tomto orientovanom stromovom grafe slová vety predstavujú vrcholy, pričom prechody medzi vrcholmi, hrany, sú reprezentované vzťahmi medzi nimi.

Ďalšia reprezentácia vzťahov zapisuje vzťahy priamo do vety. Na obrázku \ref{fig:dependencies_in_sentence} vidíme, že medzi slovami \textit{She} a \textit{looks} je vzťah \textbf{nsubj} - nominal subject, medzi \textit{looks} a \textit{beautiful} je vzťah \textbf{acomp} - adjevtival complement, a v neposlednom rade medzi slovami \textit{very} a \textit{beautiful} je vzťah \textbf{advmod} - adverb modifier.
%, ktorý je nominálnou frázou, ktorý je syntaktický subjekt vety. Nadradená časť vo vzťahu nominal subject, tomto prípade \textit{She}, nemusí byť vždy sloveso. Ak je sloveso sponové sloveso, napríklad stať sa, tak koreňom vety je doplnok sponového slovesa, ktoré môže byť prídavné meno alebo podstatné meno.

\begin{figure}[H]
\begin{center}\includegraphics[scale=0.8]{dependencies_in_sentence}\end{center}
\caption[Vzťahy vo vete]{Vzťahy vo vete}\label{fig:dependencies_in_sentence}
\end{figure}

\ifthenelse {\boolean{bachelor}}
{
	%\subsection{Subsection}
	\subsection{Nástroje na spracovanie prirodzeného jazyka}
}
{
	%\section{Subsection}
	\section{Nástroje na spracovanie prirodzeného jazyka}
}
\label{subsec:nlp_nastroje}
V súčastnosti je vyvinutých alebo sú vo vývoji viacero nástrojov, ktoré sa dajú použiť pri spracovávaní prirodzeného jazyka. Vývoj takýchto nástrojov je podporovaný na známych univerzitách ako sú napríklad Princeton, Stanford alebo Camridge ale samozrejme svoje slovo tu ma aj velikán Google. Pozrieme sa bližšie na niektoré z týchto nástrojov, čo ponúkajú a ako sa dajú využiť.

\ifthenelse {\boolean{bachelor}}
{
	%\subsection{Subsection}
	\subsubsection{WordNet}
}
{
	%\section{Subsection}
	\subsection{WordNet}
}
\label{subsubsec:wordnet}
WordNet je databáza anglických slov vyvíjaná na Princetonskej univerzite. Databáza obsahuje podstatné mena, prídavné mená, slovesá a príslovky, ktoré sú zatriedené do synonymických sád, synsetov.

Tento nástroj je dostupný vo webovej verzií (viď. Obr. \ref{fig:wordnet_search}), ale ponúka stiahnutie aj jeho databázových súborov, ktoré, po splnení licenčných požiadaviek, sa dajú využívať v projektoch.

Slová do synetov sú zaraďované podľa významu. To znamená, že slová auto a automobil, ktoré sú zameniteľné vo vete, sú zaradené do rovnakého synsetu. WordNet v súčastnosti (r. 2015) obsahuje 117 000 synsetov. Každý z týchto sysnsetov taktiež obsahuje krátku ukážku použitia slova.

Vo WordNet-e sa nachádzajú aj vzťahy medzi slovami v zmysle nadradenosti. Tým sa mysli to, že stolička je nábytok a nábytok je fyzická vec a takto to pokračuje až po najvyššie slovo, od ktorého ,,dedia'' všetky - entita (viď. Obr.  \ref{fig:wordnet_relations}. Okrem vzťahu nadradenosti WordNet obsahuje aj vzťah zloženia. Stolička sa skladá z operadla a nôh. Toto zloženie je typické len konkrétne slovo a neprenáša sa hore stromom nadradenosti,   lebo pre stoličku je typické, že sa skladá z operadla a nôh, ale to už nie je typické pre nábytok.
Prídavné mená obsahujú aj vzťah antonymity, takže slovo \textit{suchý} bude prepojené so slovom \textit{mokrý} ako so svojím antonymom.

\begin{figure}[H]
\begin{center}\includegraphics[scale=0.48]{wordnet_search}\end{center}
\caption[Webové rozhranie]{Webové rozhranie}\label{fig:wordnet_search}
\end{figure}

\begin{figure}[H]
\begin{center}\includegraphics[scale=0.48]{wordnet_search_relations}\end{center}
\caption[Nadradenosť slov]{Nadradenosť slov}\label{fig:wordnet_relations}
\end{figure}

\ifthenelse {\boolean{bachelor}}
{
	%\subsection{Subsection}
	\subsubsection{StanfordNLP}
}
{
	%\section{Subsection}
	\subsection{StanfordNLP}
}
\label{subsubsec:stanfordnlp}
Nástroj StanfordNLP je vyvíjaný na Stanfordskej univerzite. Skladá sa z niekoľkých softvérov, ktoré sa zameriavajú na úlohy spracovania prirodzeného jazyka popísané v sekcií \fullref{subsec:nlp}. Sú to softvéry \textit{Stanford Parser}, \textit{Stanford POS Tagger}, \textit{Stanford EnglishTokenizer}, \textit{Stanford Relation Extractor} a mnoho ďalších. \textit{Stanford CoreNLP} zahŕňa viacero z týchto softvérov, a práve tento nástroj budeme používať pri spracovaní učebných textov.

Nástroje StanfordNLP sú implementované v Jave, ale sú dostupné aj v iných programovacích jazykoch ako C\#, PHP alebo Python.

Dostupne je aj online webové demo. Na obrázku \fullref{fig:stanfordnlp_online_demo} vidíme výstupy z nástrojov ponúkaných balíkom StanfordNLP pre jednoduchý vstupný text skladajúci sa z jednej vety ,,Martin Nemček is student at Slovak University of Technologies in Bratislava.''.

\begin{figure}[H]
\begin{center}\includegraphics[scale=0.48]{stanfordnlp_online}\end{center}
\caption[StanfordNLP online demo]{StanfordNLP online demo}\label{fig:stanfordnlp_online_demo}
\end{figure}

\ifthenelse {\boolean{bachelor}}
{
	%\subsection{Subsection}
	\subsubsection{CambridgeAPI}
}
{
	%\section{Subsection}
	\subsection{CambridgeAPI}
}
\label{subsubsec:cambridgeapi}
CambridgeAPI je vytvorený na Cambridge univerzite. Umožňuje prístup k viacerým rôznym slovníkom. Momentálne tento nástroj ponúka prístup k pätnástim prekladovým slovníkom ako napríklad anglicko-čínsky, anglicko-ruský, anglicko-arabský, anglicko-japonský a ďalšie. Všetky prekladové slovníky majú primárny jazyk angličtinu. Slovenčinu v súčastnosti nepodporuje.

Spomínaný nástroj funguje na princípe dopytovania pomocou HTTP protokolu. Na obdržanie korektnej odpovede je potrebné mať osobný API kľúč. Ten sa dá získať kontaktovaním správcov CambridgeAPI.

\ifthenelse {\boolean{bachelor}}
{
	%\subsection{Subsection}
	\subsubsection{Google Ngram}
}
{
	%\section{Subsection}
	\subsection{Google Ngram}
}
\label{subsubsec:googlengram}
Google Ngram je postavený na ďalšom softvéry tohto giganta, Google Books. V knihách,  napísané od roku 1500 až do súčastnosti, vyhľadáva výskyty n-gramov. Podporuje len niektoré jazyky, ako angličtina, francúzština, ruština, čínština a ďalšie. Na vyhľadávanie v knihách využíva optické rozoznávanie textu, pričom dokáže spracovať regulárne výrazy, pričom tie môžu byť použité iba ako náhrada celého slova, ale nie uprostred slova. Slovné spojenie ,,* Einstein'' spracuje, pričom ,,Albert Einste*n'' nie.

N-gram je postupnosť \textit{n} za sebou idúcich slov alebo častí textu. \textit{Martin} je n-gram veľkosti jedna, 1-gram alebo unigram. \textit{Martin Nemček} je n-gram veľkosti dva, 2-gram alebo bigram a tak ďalej, pričom \textit{n} môže byť ľubovolné kladné, celé číslo.

Google Ngram Viewer poskytuje vizualizáciu vyhľadaných dát. Je dostupný vo webovom rozhraní. Na obrázku \fullref{fig:googlengram_visualization} vidno vizualizáciu výskytu mien \textit{Albert Einstein,Sherlock Holmes,Frankenstein} v knihách od roku 1800 do roku 2000.

\begin{figure}[H]
\begin{center}\includegraphics[scale=0.38]{googlengram_visualization}\end{center}
\caption[Google Ngram Viewer]{Google Ngram Viewer}\label{fig:googlengram_visualization}
\end{figure}

Tento nástroj okrem iného ponúka aj surové (angl. raw) dáta na stiahnutie.

\ifthenelse {\boolean{bachelor}}
{
	%\subsection{Subsection}
	\subsubsection{AlchemyAPI}
}
{
	%\section{Subsection}
	\subsection{AlchemyAPI}
}
\label{subsubsec:alchemyapi}
AlchemiAPI dvanásť funkcií, z ktorých sú niektoré zamerané na úlohy spracovania prirodzeného jazyka popísané v sekcií \fullref{subsec:nlp}, ako napríklad extrakcia entít, extrakcia kľúčových slov, extrakcia vzťahov, ale aj iné zaujímave funkcie, napríklad extrakcia autora z textu.

Na používanie tohto nástroja je potrebné sa zaregistrovať pre obdržanie API kľúču. S týmto kľúčom je tisíc dopytov denne zdarma. Dostupnosť v programovacích jazykoch je široká, kedže ponúka knižnicu v deviatich najpoužívanejších programovacích jazykoch.

Pre AlchemyAPI je dostupné aj online webové demo, viď obrázok \fullref{fig:alchemyapi_visualization}, kde je vidno širokú ponuku, ktorú tento nástroj ponúka.

\begin{figure}[H]
\begin{center}\includegraphics[scale=0.48]{alchemyapi_visualization}\end{center}
\caption[AlchemyAPI online demo]{AlchemyAPI online demo}\label{fig:alchemyapi_visualization}
\end{figure}

Dáta sú vo formáte JSON a okrem spracovania prirodzeného jazyka AlchemyAPI ponúka aj nástroje na extrahovanie obsahu z obrázku alebo rozpoznávanie tvári na obrázkoch.

\subsection{Enumeration}
%\subsection{Číslovaný zoznam}
\begin{my_enumerate}
	\item {goal 1}
	\begin{my_enumerate}
		\item {goal 1.a}
		\item {goal 1.b}
	\end{my_enumerate}
	\item {goal 2}
	\item {goal 3}
\end{my_enumerate}
\subsection{Itemization}
%\subsection{Zoznam}
\begin{my_itemize}
	\item {item 1}
	\begin{my_itemize}
		\item {item 1.1}
		\item {item 1.2}
	\end{my_itemize}
	\item {item 2}
	\item {item 3}
\end{my_itemize}
\subsection{Citation}
%\subsection{Citácia}
Lorem ipsum dolor sit amet, consectetuer adipiscing elit, sed diam nonummy nibh euismod tincidunt ut laoreet dolore magna aliquam erat volutpat~\cite{1}.

\subsection{Labesl \& References}
%\subsection{Návestia \& Referencie}
See Section~\ref{lab:Examples}.

\subsection{Examples}
%\subsection{Príklady}
\label{lab:Examples}

\begin{lstlisting}[ language=html, caption={Example 1}, label={metrics_LOC},
	keywordstyle=\color{blue}\bfseries,
	ndkeywordstyle=\color{black}\bfseries,
	commentstyle=\color{red}\ttfamily,
	stringstyle=\color{green}\ttfamily,
	identifierstyle=\color{gray},
	backgroundcolor=\color{white}, 
	frame=single, 
	frameround=ffff,
	captionpos=b,
	basicstyle=\scriptsize
	]
<table class="metric_index">
	<tr>
		<th>Lines of code</th>
		<th>Value</th>
	</tr>
	<% if (filenum and modulenum) then %>
		<tr>
			<td class="name">Number of files</td>
			<td class="value"><%=filenum%></td>
		</tr>
		<tr>
			<td class="name">Number of modules</td>
			<td class="value"><%=modulenum%></td>
		</tr>
		<tr>
	<% end %>
	<tr>
		<td class="name">Lines Total</td>
		<td class="value"><%=LOC.lines%></td>
	</tr>
	<!--
							skryty zdrojovy kod
		podobne zobrazenie ostatnych metrik riadkov
	-->
</table>
\end{lstlisting}

\begin{lstlisting}[language=lua, caption={Názov}, label=metrics.pipe]
local parser  = require 'leg.parser'
local rules = require 'metrics.rules'
-- << skryty zdrojovy kod >> --
local capture_table = {}
grammar.pipe(LOC_capt.captures, AST_capt.captures)
grammar.pipe(block_capt.captures, LOC_capt.captures)
-- << viacero rovnakych volani s tabulkami captures inych modulov >> --
grammar.pipe(capture_table, cyclo_capt.captures)
local lua = lpeg.P(grammar.apply(parser.rules, rules.rules, capture_table))
local patt = lua / function(...) 
	return {...} 
end
local result = patt:match(code)[1]
\end{lstlisting}
