\newpage
%
% Návrh
%
\ifthenelse {\boolean{bachelor}}
{
	%\section{Design}
	\section{Návrh}
}
{
	%\chapter{Design}
	\chapter{Návrh}
}
\label{section:design}

%
% Návrh uchovávania textov v databázach
%
\ifthenelse {\boolean{bachelor}}
{
	%\subsection{Subsection}
	\subsection{Návrh uchovávania textov v databázach}
}
{
	%\section{Subsection}
	\section{Návrh uchovávania textov v databázach}
}
\label{subsection:our_design_persisting_data}
Dáta budeme ukladať v dokumentovej databáze MongoDB. Keďže spracovávané dáta sa dajú rozdeliť do troch kategórií, budeme využívať primárne tri databázové kolekcie na ich ukladanie. Sú to:

\begin{my_itemize}
	\myitem sentences,
	\myitem rules,
	\myitem texts.
\end{my_itemize}
	
Pri návrhu sme vychádzali z princípu čo najjednoduchších kolekcií, ktoré budu obsahovať iba relevantné informácie.
V nasledujúcich častiach ich opíšeme bližšie aj s názornými ukážkami.

%
% Kolekcia texts
%
\ifthenelse {\boolean{bachelor}}
{
	%\subsection{Subsection}
	\subsubsection{Kolekcia texts}
}
{
	%\section{Subsection}
	\subsection{Kolekcia texts}
}
V kolekcií \textit{texts} sa ukladajú celé texty, ktoré sú spracovávané. 

Kolekcia obsahuje iba jedno pole textového typu slúžiace na uloženie textu v pôvodnom tvare. Štruktúra uložených dát v kolekcií \textit{texts} je zobrazená na obrázku~\fullref{fig:texts_collection_structure}.

\begin{figure}[H]
	\begin{center}\includegraphics[scale=0.60]{texts_collection}\end{center}
	\caption[Štruktúra kolekcie texts]{Štruktúra kolekcie texts}\label{fig:texts_collection_structure}
\end{figure}

%
% Kolekcia sentences
%
\ifthenelse {\boolean{bachelor}}
{
	%\subsection{Subsection}
	\subsubsection{Kolekcia sentences}
}
{
	%\section{Subsection}
	\subsection{Kolekcia sentences}
}
V ďalšej kolekcii \textit{sentences} ukladáme spracovávané vety a vytvorené poznámky z týchto viet, pričom vety sa odkazujú na texty, z ktorých pochádzajú v kolekcií \textit{texts}. Umožní nám to jednoducho zistiť, v akom texte sa daná veta nachádzala.

Dáta sú uložené v dokumentoch, ktoré obsahujú tri polia. Jedno, textové, určené na uchovanie pôvodného znenia vety, druhé tiež textové na uchovanie novo vytvorenej vety po spracovaní vety uloženej v prvom poli a tretie pole, ktoré bude odkazovať na záznam v kolekcii \textit{texts}. Štruktúra dát v tejto kolekcií je načrtnutá na obrázku~\fullref{fig:sentences_collection_structure}.

\begin{figure}[H]
	\begin{center}\includegraphics[scale=0.60]{sentences_collection}\end{center}
	\caption[Štruktúra kolekcie sentences]{Štruktúra kolekcie sentences}\label{fig:sentences_collection_structure}
\end{figure}

%
% Kolekcia rules
%
\ifthenelse {\boolean{bachelor}}
{
	%\subsection{Subsection}
	\subsubsection{Kolekcia rules}
}
{
	%\section{Subsection}
	\subsection{Kolekcia rules}
}
V poslednej kolekcii pomenovanej \textit{rules} sa ukladajú pravidla na spracovávanie viet, ktoré sa odkazujú na vety v kolekcii \textit{sentences}, ktoré boli podľa daného pravidla spracované. Ukladaním viet a pravidiel na ich spracovanie do separátnych kolekcií zabránime duplikovaniu dát a zrýchlime vyhľadávanie. Referencia do kolekcie \textit{sentences} nám poskytuje možnosť jednoduchého a rýchleho vyhľadanie viet, na ktoré bolo konkrétne pravidlo aplikované a aký bol výstup aplikovania tohto pravidla.

Pravidlo sa skladá hlavne z dvoch častí. Zoznam závislostí pôvodnej vety a zoznam závislostí zjednodušenej vety. Práve závislosti z druhého menovaného zoznamu sa aplikujú na spracovávanú vetu s cieľom zjednodušiť ju.

Každý záznam v tejto kolekcii obsahuje pole celých čísel určujúcich pozície slov, za ktorými je vo vytvorenej zjednodušenej vete ukončenie vety. V prípade jednoduchých viet to bude posledné slovo vety, ale pri súvetiach to môže byť viacero slov na ľubovolných miestach vety. Pre jednoduchú vetu \textit{,,The president of the Czech Republic is Miloš Zeman.''} bude toto pole obsahovať hodnotu 3, keďže zjednodušená veta bude v tvare \textit{,,President is Zeman.''}. Pre zloženú vetu v tvare \textit{,,Czech Republic has no sea; its neighbour countries are Germany, Austria, Slovakia and Poland.''} bude spomínané pole obsahovať dve hodnoty, keďže táto veta sa skladá z dvoch. Prvá obsahujúca informáciu o mori a druhá s informáciou o susedných štátoch, a tak sa aj spracuje pri zjednodušovaní.

Okrem poľa určujúceho konce viet, bude každý záznam obsahovať dva hlavné zoznamy závislostí. Prvý zoznam bude pozostávať zo závislostí pôvodnej vety a druhý zoznam bude zložený zo závislostí zjednodušenej vety. Zoznamy majú nasledujúcu štruktúru. Tieto dokumenty majú názov vzťahu závislosti a ich zoznam, pričom sa párujú práve podľa názvu. Tento vnorený zoznam obsahuje už konkrétne závislosti. Každá závislosť uložená v databáze sa skladá z nadradeného tokenu (angl. governor), podradeného tokenu (angl. dependent) a pozície tejto závislosti medzi všetkými závislosťami vety. Tokeny sú dokumenty skladajúce sa z dvoch polí, jedno textové, obsahujúce skratku POS značky a druhé číselne, obsahujúce pozíciu slova vo vete, ku ktorému sa daný token viaže.

Celá štruktúra dát v kolekcii \textit{rules} sa dá vyjadriť diagramom~\fullref{fig:rules_collection_structure}.

\begin{figure}[H]
	\begin{center}\includegraphics[scale=0.45]{rules_collection}\end{center}
	\caption[Štruktúra kolekcie rules]{Štruktúra kolekcie rules}\label{fig:rules_collection_structure}
\end{figure}

Dáta sú v MongoDB databáze uložené v binárnom JSON formáte. Na ukážke~\fullref{code:collection_rules_data_example} je zobrazená časť uložených údajov o pôvodnej vete. Ukážka celého záznamu pre vetu ,,The president of the Czech Republic is Milos Zeman.'' je priložená v prílohe~\fullref{appendix:db_entry_full_example}.
\\
\begin{lstlisting}[language = json, caption={Ukážka dát kolekcie rules}, label = {code:collection_rules_data_example}]
{  
	"originalDependencies" : [  
		{  
			"dependencyName" : "det",
			"dependencies" : [  
				{  
					"governor" : {  
						"pos" : "NN",
						"index" : 2
					},
					"dependent" : {  
						"pos" : "DT",
						"index" : 1
					},
					"position" : 0
				},
				{ ... }
			]
		}
	]
}
\end{lstlisting}


%
% Manažment dát
%
\ifthenelse {\boolean{bachelor}}
{
	%\subsection{Subsection}
	\subsection{Manažment dát}
}
{
	%\section{Subsection}
	\section{Manažment dát}
}
\label{subsection:data_management}
V nasledujúcich častiach si priblížime prácu s dátami z databázy, ako vyhľadanie pravidla, jeho aplikovanie alebo vytvorenie pravidla, ak žiadne nebolo vyhľadané.

%
% Vyhľadávanie pravidla
%
\ifthenelse {\boolean{bachelor}}
{
	%\subsection{Subsection}
	\subsubsection{Vyhľadávanie pravidla}
}
{
	%\section{Subsection}
	\subsection{Vyhľadávanie pravidla}
}
\label{subsubsection:rule_lookup}
Pred spracovaním vety sa vyhľadá pravidlo v databáze vhodné na jej zjednodušenie. Pri vyhľadávaní sa berie do úvahy viacero podmienok.

Spracovávaná veta, pre ktorú hľadáme pravidlo, musí mať rovnaký počet záznamov v \textit{zozname závislosti pôvodnej vety} a zároveň musia byť napárované práve všetky názvy vzťahov v závislostiach v tomto zozname.

Pri použití týchto podmienok vieme rýchlo vyhľadať pravidlo, ktoré súvisí s podobnou vetou. Avšak, môže nastať situácia, kedy je pre spracovávanú vetu vhodných viacero pravidiel. Vtedy sa rozhoduje podľa zhody pôvodných viet, ktoré vybrať. Vyberá sa, a následne aplikuje, to s najväčšou zhodou.

Určovanie najväčšej zhody má viacero krokov. Najskôr sa spočítavajú zhody POS značiek nadradených a podradených tokenov zvlášť a následne, indexy slov prislúchajúcich tokenom taktiež nezávisle od seba. Tým sa zisťuje, či spracovávaná veta obsahuje ľubovoľnú závislosť s rovnakou hodnotou POS značky alebo indexu či už nadradeného alebo podradeného tokenu. V druhom kroku sa určuje polovičná zhoda závislosti, teda či spracovávaná veta obsahuje zhody POS značiek a zároveň indexov slov v nadradenom tokene alebo v podradenom tokene. V poslednom, treťom sa zisťuje počet úplných zhôd závislostí, čo znamená zhoda POS značiek a indexov zároveň, v nadradenom a podradenom tokene zároveň. Tieto tri hodnoty sa na záver spočítajú a tým získame percentuálne ohodnotenie zhody viet.

Toto určovanie najväčšej zhody sa uskutoční pre každé vyhovujúce pravidlo a vyberie sa pravidlo s najväčšou zhodou. \\

Predpokladajme situáciu kedy spracovávame vetu ,,The local language is Czech language.'' a v databáze máme, okrem iného, uložené pravidlá pre vety ,,The local language is Czech language'' a ,,The Czech language is a Slavic language.''. Vtedy nastane, že pre spracovávanú vetu je vhodných viacero pravidiel. Keďže obe vety majú v \textit{zozname závislostí pôvodnej vety} práve 5 záznamov a tieto záznamy sa skladajú práve z množiny vzťahov \{det, amod, nsubj, cop, root\}, vyhľadanie pravidlá nám vráti minimálne tieto dve pravidlá. Po aplikovaní určenia najväčšej zhody týchto dvoch pravidiel a spracovávanou vetou, zistíme, že pravidlo prvej menovanej vety má so spracovávanou vetou cca. $99,99\%$ zhodu a pravidlo druhej vety má cca. $63,57\%$ zhodu. Aplikuje sa prvé pravidlo.

Ukážkový proces vyhľadania pravidla a určenie zhody je zobrazený na obrázku~\fullref{fig:rule_lookup}.

\begin{figure}[H]
	\begin{center}\includegraphics[scale=0.6]{rule_lookup}\end{center}
	\caption[Vyhľadanie pravidla]{Vyhľadanie pravidla}\label{fig:rule_lookup}
\end{figure}

%
% Vytváranie pravidla
%
\ifthenelse {\boolean{bachelor}}
{
	%\subsection{Subsection}
	\subsubsection{Vytváranie pravidla}
}
{
	%\section{Subsection}
	\subsection{Vytváranie pravidla}
}
\label{subsubsection:rule_creation}
Ak nám proces vyhľadania pravidla nevyhľadal žiadne pravidlo, znamená to, že sme doposiaľ nespracovávali takú istú alebo podobnú vetu. V tomto prípade použijeme náš parser, ktorý operuje nad staticky danou sadou pravidiel. Výstupom parseru bude zjednodušená veta, ktorej pravidlo sa následne uloží do databázy a pri ďalšom spracovávaní takej istej alebo podobnej vety sa toto pravidlo vyhľadá a aplikuje ak bude mať dostatočné veľkú zhodu.

Zo závislostí pôvodnej vety sa vytvorí \textit{zoznam závislostí pôvodnej vety}, zo závislostí zjednodušenej vety sa vytvorí \textit{zoznam závislostí zjednodušenej vety} a zo zjednodušenej vety sa určia konce viet. Tieto informácie sa spolu uložia do dokumentu (záznamu) do databázy ako \textbf{pravidlo}.

Na obrázku~\fullref{fig:rule_creation} je znázornený proces nevyhľadania pravidla, použitia parsera s následným uložením nového pravidla.

\begin{figure}[H]
	\begin{center}\includegraphics[scale=0.5]{rule_creation}\end{center}
	\caption[Vytvorenie pravidla]{Vytvorenie pravidla}\label{fig:rule_creation}
\end{figure}

%
% Aplikovanie pravidla
%
\ifthenelse {\boolean{bachelor}}
{
	%\subsection{Subsection}
	\subsubsection{Aplikovanie pravidla}
}
{
	%\section{Subsection}
	\subsection{Aplikovanie pravidla}
}
\label{subsubsection:rule_application}
Máme spracovávanú vetu a pravidlo na aplikovanie. Z princípu vyhľadávania pravidiel (viď.~\fullref{subsubsection:rule_lookup} pre podrobnosti) vieme, že spracovávaná veta obsahuje závislosti zo zoznamu závislostí pôvodnej vety a tým pádom obsahuje aj závislosti zo zoznamu závislostí zjednodušenej vety. 

Proces aplikovania pravidla prebieha nasledovne. Pre každú závislosť v zozname závislostí zjednodušenej vety, sa vyhľadá táto závislosť v spracovávanej vete. Z nájdenej závislosti sa zoberie slovo prislúchajúce podradenému tokenu a pridá sa do výslednej zjednodušenej vety na miesto svojho indexu. V prípade ak sa jedná o závislosť \textit{nominal subject}, zoberie sa aj slovo prislúchajúce nadradenému tokenu a taktiež sa pridá do výslednej vety. Po prejdení všetkých závislostí zo zoznamu sa urobia posledné úpravy zjednodušenej vety, ako kapitalizácia prvého písmena vety a rozdelenie vety na viacero viet, ak tak definovalo pravidlo.

Funkcia, ktorá bude aplikovať pravidlo na vetu s cieľom získania zjednodušenej vety bude vyzerať ako je naznačené v ukážke~\fullref{code:apply_rule_example}.
\\

\begin{lstlisting}[language = csharp, caption={Aplikovanie pravidla}, label = {code:apply_rule_example}]
// Funkcia aplikuje pravidlo na vetu a vrati zjednodusenu vetu
private Note ApplyRule(NotenizerSentence sentence, NotenizerRule rule)
{
	// Vytvorenie objektu zjednodusenej vety
	Note note = new Note(sentence);
	// Vytvorenie objektu casti zjednodusenej vety - zjedn. veta sa moze skladat
	// z viacerych
	NotePart notePart = new NotePart(sentence);

	foreach (NotenizerDependency dependencyLoop in rule.RuleDependencies)
	{
		// Vyhlada zavislost a prida hodnotu prisluchajuceho slova do vyslednej
		// zjednodusenej vety
		ApplyRulesDependency(sentence, dependencyLoop, notePart);
	}

	// Finalne upravy
	note.Add(notePart);
	note.SplitToSentences(rule.SentencesEnds);

	return note;
}
\end{lstlisting}

Pre vetu ,,The president of the Czech Republic is Miloš Zeman.'' nám nástroj Stanford CoreNLP poskytne závislosti slov vo vete, ktoré sú v textovej podobe výstupu zobrazené v ukážke~\fullref{code:sentence_dependencies}. Závislosti sú v tvare\\ 

[názov závislosti] ([slovo prislúchajúce nadradenému tokenu] - [index slova vo vete], [slovo prislúchajúce podradenému tokenu] - [index slova vo vete]).\\

Ak na túto vetu aplikujeme pravidlo, ktoré obsahuje v zozname závislostí zjednodušenej vety dve závislosti:
\begin{enumerate}
	\item nsubj
	\begin{itemize}
		\item podradený token
		\begin{itemize}
			\item POS: NN
			\item index: 2
		\end{itemize}
		\item nadradený token
		\begin{itemize}
			\item POS: NNP
			\item index: 9
		\end{itemize}
	\end{itemize}
	
	\item cop
	\begin{itemize}
		\item podradený token
		\begin{itemize}
			\item POS: VBZ
			\item index: 7
		\end{itemize}
		\item nadradený token
		\begin{itemize}
			\item POS: NNP
			\item index: 9
		\end{itemize}
	\end{itemize}
\end{enumerate}

Tak výsledná zjednodušená veta bude ,,President is Zeman.''. \\

\begin{lstlisting}[language = csharp, caption={Závislosti jednoduchej vety}, label = {code:sentence_dependencies}]
root(ROOT-0, Zeman-9)
det(president-2, The-1)
nsubj(Zeman-9, president-2)
case(Republic-6, of-3)
det(Republic-6, the-4)
compund(Republic-6, Czech-5)
nmod:of(president-2, Republic-6)
cop(Zeman-9, is-7)
compund(Zeman-9, Milos-8)
\end{lstlisting}