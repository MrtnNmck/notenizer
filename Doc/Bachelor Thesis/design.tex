\newpage
%
% Návrh
%
\ifthenelse {\boolean{bachelor}}
{
	%\section{Design}
	\section{Návrh}
}
{
	%\chapter{Design}
	\chapter{Návrh}
}
\label{section:design}

%
% Uchovávanie textov v databázach
%
\ifthenelse {\boolean{bachelor}}
{
	%\subsection{Subsection}
	\subsection{Uchovávanie textov v databázach}
}
{
	%\section{Subsection}
	\section{Uchovávanie textov v databázach}
}
\label{subsection:persisting_texts_in_db}
Text je špecifický údajový model s variabilnou štruktúrou. Ak chceme efektívne ukladať texty v databázach, je nutné aby sme použili databázu, ktorá je tomu prispôsobená, pri ktorej nebudeme zbytočne čerpať pamäť a takisto bude jednoduché narábať s dátami. To znamená bezproblémové ukladanie, získavanie, vyhľadávanie a spracovanie textov na úrovni databázy. V nasledujúcich kapitolách sa pozrieme, aké typy databáz existujú a aké možnosti z pohľadu ukladania textov ponúkajú.

%
% Relačné databázy
%
\ifthenelse {\boolean{bachelor}}
{
	%\subsection{Subsection}
	\subsubsection{Relačné databázy}
}
{
	%\section{Subsection}
	\subsection{Relačné databázy}
}
\label{subsubsection:relation_dbs}
Relačné databázy boli dlhé roky populárnou a finančné nenáročnou voľbou pri tvorbe veľkých podnikateľských aplikácií~\cite{MongoDBvsMySQLCompared}. Relačné databázy sú momentálne používané vo väčšine súčasných aplikácií a pracujú spoľahlivo pri obmedzenom množstve dát~\cite{MongoDBvsMySQL2015}. Problém s relačným modelom relačných databáz nastáva, keď vzniká potreba aplikácie s obrovským množstvom dát. Menovite rozšíriteľnosť (angl. scalability) sa stáva najväčším problémom relačných databáz~\cite{NoSQLDBvsRealtionDB}.

%
% Textové databázy
%
\ifthenelse {\boolean{bachelor}}
{
	%\subsection{Subsection}
	\subsubsection{Textové databázy}
}
{
	%\section{Subsection}
	\subsection{Textové databázy}
}
\label{subsubsection:text_dbs}
S rozmachom variácie dát v posledných rokoch sa začali objavovať a vznikať nerelačné databázy, aby pokryli požiadavky na nové aplikácie.~\cite{MongoDBvsMySQLCompared}. Textové databázy sú druhom nerelačných databáz.

Textové databázy ukladajú dáta vo forme dokumentov, vďaka čomu ponúkajú vysoký výkon a horizontálnu rozšíriteľnosť~\cite{NoSQLDBvsRealtionDB}. Uložené dokumenty môžu nadobúdať rôzne typy, ako napríklad JSON, BSON, XML a BLOB, ktoré poskytujú veľkú flexibilnosť pre dáta~\cite{MongoDBvsMySQLCompared}. Každý záznam v takejto databáze preto môže mať inú štruktúru, napríklad počet alebo typ polí, čo šetrí úložným priestorom, keďže neobsahuje nepotrebné prázdne polia~\cite{NoSQLDBvsRealtionDB}.

Dokumenty v databáze sú referencované kľúčom, ktorý môže byť string, cesta, ale dokonca aj dokument~\cite{NoSQLDBvsRealtionDB}. Majú dynamickú schému, čo umožňuje vytvárať záznamy bez toho, aby bolo potrebné predtým definovať štruktúru. Uľahčujú zmenu štruktúry záznamov jednoduchým pridaním, odstránením alebo zmenením typu poľa. Vďaka svojej štruktúre sú dokumenty ľahko namapovateľné na objekty z objektovo-orientovaných programovacích jazykov a odstraňujú tým potrebu pre použitie objektovo-relačnej mapovacej vrstvy~\cite{MongoDBvsMySQLCompared}.
\\

Primárne využitie týchto databáz je v aplikáciách, ktoré potrebujú ukladať dáta, ktorých štruktúra je vopred neznáma alebo sa mení. Predstaviteľmi sú napríklad \textit{MongoDB} alebo \textit{CouchDB} databázy.

%
% MongoDB
%
\ifthenelse {\boolean{bachelor}}
{
	%\subsection{Subsection}
	\paragraph{MongoDB}
}
{
	\%section{Subsection}
	\subsubsection{MongoDB}
}
\label{subsection:mongodb}
MongoDB je dokumentová nerelačná databáza vytvorená v C++ spustená v roku 2009~\cite{NoSQLDBvsRealtionDB}. Ukladá dáta v dokumentoch vo formáte BSON (Binary JSON), ktorých štruktúra sa môže meniť. Využíva dynamickú štruktúru schém, preto dokáže vytvárať záznamy bez preddefinovanej štruktúry dát, lebo štruktúra sa vytvára za behu, pričom môže byť veľmi jednoducho pozmenená pridaním, odstránením, zmenou typu polia dokumentu určujúceho štruktúru. Umožňuje jednoduché ukladanie dát s hierarchickými vzťahmi alebo komplexnejších štruktúr, ako sú napríklad polia, listy alebo vnorené polia~\cite{MongoDBvsMySQLCompared}.

Vlastnosti ako chybová tolerancia, perzistencia a konzistencia dát sú súčasťou MongoDB. Oproti klasickým dokumentovým databázam ponúka aj vymoženosti, ako agregácia, ad hoc dopyty, indexovanie, a pod. Taktiež má svoj vlastný plnohodnotný dopytovací jazyk \textit{mongo query language}~\cite{NoSQLDBvsRealtionDB}.

Prvky poskytované databázou MongoDB sú prvky zahrnuté v relačných databázach rozšírené o ďalšiu funkcionalitu. Porovnanie poskytovaných prvkov je v tabuľke~\fullref{table:features_of_mongodb}. 

\begin{table}[H]
	\centering
	\caption{Prvky poskytované MongoDB~\cite{MongoDBvsMySQLCompared}}
	\label{table:features_of_mongodb}
	\begin{tabular}{|l|l|l|}
		\hline
		& \textbf{MySQL} & \textbf{MongoDB} \\ \hline
		Bohatý dátový model & Nie & Áno \\ \hline
		Dynamická štruktúra & Nie & Áno \\ \hline
		Dátové typy & Áno & Áno \\ \hline
		Lokálnosť dát & Nie & Áno \\ \hline
		Aktualizovanie polí & Áno & Áno \\ \hline
		Ľahké pre programátorov & Nie & Áno \\ \hline
		Komplexné transakcie & Áno & Nie \\ \hline
		Kontrola & Áno & Áno \\ \hline
		Auto-sharding & Nie & Áno \\ \hline
	\end{tabular}
\end{table}

MongoDB má vlastnú konvenciu názvov svojich častí. Tie sa v niektorých prípadoch líšia s názvami relačných databáz. Rozdiely sú zobrazene v tabuľke~\fullref{table:names_of_mongodb}. Za zástupcu relačných databáz bola vybraná MySQL databáza. 

\begin{table}[H]
	\centering
	\caption{Porovnanie konvencie názvov~\cite{MongoDBvsMySQL2015}}
	\label{table:names_of_mongodb}
	\begin{tabular}{|l|l|}
		\hline
		\textbf{MySQL} & \textbf{MongoDB} \\ \hline
		Databáza & Databáza \\ \hline
		Tabuľka & Kolekcia \\ \hline
		Index & Index \\ \hline
		Riadok & BSON dokument \\ \hline
		Stĺpec & BSON pole (angl. field) \\ \hline
		Spojenie & Vnorené dokumenty a prepojenie \\ \hline
		Primárny kľúč & Primárny kľúč \\ \hline
		Zoskupenie & Agregácia \\ \hline
	\end{tabular}
\end{table}

%
% Ostatné
%
\ifthenelse {\boolean{bachelor}}
{
	%\subsection{Subsection}
	\subsubsection{Ostatné}
}
{
	\%section{Subsection}
	\subsection{Ostatné}
}
\label{subsection:types_of_norelation_dbs}
Okrem relačných a textových dokumentov existuje ešte niekoľko druhov databáz. V nasledujúcich častiach si priblížime niektoré z nerelačných databáz.

%
% Kľúč - hodnota databázy
%
\ifthenelse {\boolean{bachelor}}
{
	%\subsection{Subsection}
	\paragraph{Kľúč - hodnota databázy}
}
{
	\%section{Subsection}
	\subsubsection{Kľúč - hodnota databázy}
}
\label{subsubsection:key_value_db}
Nerelačné databázy typu kľúč - hodnota sú v svojej podstate celkom jednoduché, ale zároveň efektívne. Umožňujú používateľovi ukladať dáta ľubovolne, kedže neobsahujú schémy. Uložené dáta sa skladajú z dvoch častí. Prvá časť je kľuč a druhá časť je hodnota~\cite{NoSQLDBvsRealtionDB}, pričom kľúč je samo-generujúci string a hodnota môže byť takmer čokoľvek, od string, JSON cez BLOB až po obrázok~\cite{MongoDBvsMySQL2015}.

Kľúč - hodnota databázy sú veľmi podobné hašovacím tabuľkám, kde kľúč je indexom do tabuľky, pomocou ktorého používateľ môže pristúpiť k hodnote daného kľúču. Tento typ databáz uprednostňuje rozšíriteľnosť pred konzistenciou. Ponúka vysokú konkurenčnosť (angl. concurrency), rýchle vyhľadávanie a schopnosť uloženia veľkého množstva dát za cenu spojovacích a agregačných operácií. Taktiež je veľmi náročné vytvoriť ľubovolný pohľad na dáta z dôvodu chýbajúcej schémy~\cite{NoSQLDBvsRealtionDB}.

Najznámejšími predstaviteľmi tohto typu databáz sú \textit{Amazon DynamoDB} a \textit{RIAK}.

%
% Stĺpcové databázy
%
\ifthenelse {\boolean{bachelor}}
{
	%\subsection{Subsection}
	\paragraph{Stĺpcové databázy}
}
{
	\%section{Subsection}
	\subsubsection{Stĺpcové databázy}
}
\label{subsubsection:column_db}
Stĺpcové databázy musia mať preddefinovanú schému, v ktorej sú jednotlivé bunky záznamov zoskupené do kolekcie stĺpcov~\cite{MongoDBvsMySQL2015}. Dáta nie su ukladané do tabuliek, ale do masívne distribuovaných architektúr, za hlavným zámerom, aby agregácia dát mohla prebehnúť veľmi rýchlo s redukovaním I/O aktivity.

Tento typ databáz taktiež poskytuje veľkú rozšíriteľnosť v ukladaní dát.

Najvhodnejšie je využívať takéto databázy v analytických aplikáciách alebo aplikáciach, ktoré získavajú dáta pomocou metódy \textit{data mining}~\cite{NoSQLDBvsRealtionDB}.

%
% Grafové databázy
%
\ifthenelse {\boolean{bachelor}}
{
	%\subsection{Subsection}
	\paragraph{Grafové databázy}
}
{
	\%section{Subsection}
	\subsubsection{grafové databázy}
}
\label{subsubsection:graph_db}
Grafové databázy su špeciálny typ databáz, v ktorých sú dáta uložene vo forme grafu. Graf pozostáva z vrcholov a hrán, pričom vrcholy predstavujú objekty a hrany reprezentujú vzťahy medzi nimi. Každý vrchol okrem iného obsahuje aj ukazovateľ na priľahlé vrcholy, čo umožňuje prechádzať obrovské množstvo dát rýchlejšie ako v relačných databázach~\cite{NoSQLDBvsRealtionDB}.

Údaje sa ukladajú v polo-štruktúrovanej forme, kde je kladený hlavný dôraz na prepojenia medzi dátami. Grafové databázy spĺňajú vlastnosť ACID a sú veľmi vhodné pre biometrické aplikácie alebo aplikácie sociálnych sieti. Hlavným predstaviteľom grafových databáz je \textit{Neo4j}~\cite{NoSQLDBvsRealtionDB}.

%
% Objektovo orientované databázy
%
\ifthenelse {\boolean{bachelor}}
{
	%\subsection{Subsection}
	\paragraph{Objektovo orientované databázy}
}
{
	\%section{Subsection}
	\subsubsection{Objektovo orientované databázy databázy}
}
\label{subsubsection:object_oriented_db}
Objektovo orientované databázy ukladajú dáta vo forme objektov, rovnako ako sú údaje reprezentované v objektoch v objektovo orientovaných programovacích jazykoch (OOP). Tieto databázy podporujú všetky vymoženosti OOP, ako enkapsulácia, polymorfizmus, ale aj dedenie. Objektovo orientované databázy robia moderný vývoj softvéru jednoduchším~\cite{NoSQLDBvsRealtionDB}.

%
% Zhrnutie
%
\ifthenelse {\boolean{bachelor}}
{
	%\subsection{Subsection}
	\subsubsection{Zhrnutie}
}
{
	%\section{Subsection}
	\subsection{Zhrnutie}
}
NOSQL databáza narozdiel do RDBMS modelu (Relation Data Base Management System) je
navrhnutá aby sa bola jednoducho rozšíriteľná so zväčšovaním sa. Väčšina NOSQL databáz odstránila niektoré nepotrebné prvky RDBMS modelov, čím sa stali podstatne ľahšími a efektívnejšími ako ich náprotivok RDBMS systémy. Toto na druhej strane spôsobilo, že NOSQL model negarantuje vlastnosti ACID (Atomicity, Consistency, Isolation, Durability), ale naopak garantuje vlastnosti BASE (Basically Available, Soft state, Eventula Consistency)~\cite{NoSQLDBvsRealtionDB}.

Nerelačné databázy neukladajú údaje v tabuľkách, nemajú fixnú schému a majú jednoduchý dátový model. Tieto vlastnosti im umožňujú jednoducho spracovávať neštrukturované dáta, ako sú dokumenty, e-maily a mnoho ďalších~\cite{MongoDBvsMySQL2015}. Tieto databázy majú čím ďalej, tým majú viacero využití.

Existuje hneď niekoľko prípadov, kedy je lepšie použiť nerelačnú databázu namiesto relačnej databázy. Keď je potrebné, aby aplikácia dokázala spracovávať rôzne typy a tvary dát alebo pri potrebe spravovať aplikáciu efektívnejšie pri rozširovaní, je rozhodne výhodnejšie použiť nerelačnú databázu. Niektoré databázy, ako napríklad textová databáza MongoDB uľahčuje vývoj aplikácií, keďže jeho dokumentová štruktúra dát je jednoducho namapovateľná na moderné, objektovo-orientované programovacie jazyky a tým pádom nie je potreba využívať komplexnú objektovo-relačnú mapovaciu vrstvu, ktorá je nutná pri použití relačných databáz na prevod objektov z programovacie jazyka na perzistentné objekty v databáze. Všeobecne je omnoho ľahšie rozšíriť schému / model nerelačnej databázy ako rozširovať schému relačnej databázy~\cite{MongoDBvsMySQLCompared}.

My budeme využívať textovú databázu, konkrétne MongoDB, na ukladanie spracovaných dát. Keďže skoro každý text a veta je odlišná, tak pre každý text a vetu budeme ukladať, odlišné alebo odlišný počet dát, takže nebudeme mať fixne danú schému. Kvôli tomu, ale aj pre ostatné vlastnosti dokumentovej databázy MongoDB, podrobnejšie opísané v kapitole~\fullref{subsection:mongodb}, bola voľba tejto databázy jednoznačná.