\newpage
\ifthenelse {\boolean{bachelor}}
{
	%\section{Introduction}
	\section{Úvod}
}
{
	%\chapter{Introduction}				
	\chapter{Úvod}
}
Internet je v dnešných dňoch zaplnený obrovským množstvom dát a informácií. Mnohé z týchto dát sa na internete vyskytujú mnohonásobne, či už v identickej podobe alebo s úpravou. Avšak, čím ďalej tým viac z týchto informácií vyskytujúcich sa na internete, sú informácie irelevantné.

Stáva sa to až príliš často a každý už zažil situáciu, kedy hľadal informácie na konkrétnu tému a musel sa ,,prehrabať'' kopou nepodstatných dát a informácií, ktoré mu boli ponúkané. Stáva sa to medzi všetkými kategóriami používateľov na internete.

Jednou z majoritných kategórií používateľov, ktorí sa s takouto situáciou stretávajú denno denne sú študenti. Študenti všetkých škôl, od základných až po univerzity, získavajú informácie na učenie, projekty alebo zadania primárne z internetu alebo učebných textov kníh. Keď musia prechádzať obrovské množstvá dát z rôznych zdrojov, je to náročne, často až frustrujúce a berie im to veľmi veľa času. Tento čas by mohli využiť efektívnejšie, napríklad na rozvoj svojich vedomostí.

Učebné texty sú často písané v neštruktúrovanej forme a prirodzenom jazyku. Pre stroje je mnohokrát náročné správne interpretovať tieto informácie. Jedným z hlavných dôvodov je fakt, že každý jazyk je odlišný a obsahuje špecifické charakteristiky, ktoré môžu byť napríklad slovosled vety, gramatické kategórie slov, ale aj vetné členy a vzťahy medzi nimi.

Tieto, ale aj mnohé iné charakteristiky jazyka sa dajú využiť pri jeho spracovaní a reprezentáciu do podoby, s ktorou vedia aj stroje jednoducho narábať. Takýto proces sa nazýva \textit{spracovanie prirodzeného jazyka} (angl. Natural Language Processing - NLP). Spracovanie prirodzeného jazyka má viacero aplikácií, z ktorých sú to napríklad preklad jazyka, vytiahnutie najpodstatnejších entít z textu, prípadne aj štatistika ich výskytu a mnohé ďalšie.

My posunieme spracovanie prirodzeného jazyka ešte o kúsok ďalej a budeme sa zaoberať ako dopomôcť študentom so spracovaním veľkého množstva informácií, hlavne z učebných textov. Študentom najviac pomôže, ak dokážu rýchlo z textu vytiahnuť tie najpodstatnejšie, najdôležitejšie informácie a údaje, ktoré sa im ďalej budú omnoho ľahšie spracovávať a učiť. Proces určovania a extrakcie najpodstatnejších informácií z učebného textu môžme nazvať spoznámkovávanie.

Zameriame sa hlavne na využitie vetných členov a vzťahov medzi nimi, na určenie najpodstatnejšej, najrelevantnejšej informácie z vety. Takto extrahované informácie následne ponúkneme používateľovi (študentovi).