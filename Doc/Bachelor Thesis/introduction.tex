\newpage
\ifthenelse {\boolean{bachelor}}
{
	%\section{Introduction}
	\section{Úvod}
}
{
	%\chapter{Introduction}				
	\chapter{Úvod}
}
V dnešnej dobe sme obklopení veľkým množstvom dát z rôznych zdrojov, či už internetu, kníh alebo iných. Vo vzdelávaní je problém vytvoriť poznámky zo zdroja, ktoré by zahrňovali podmnožinu dôležitých informácií. 

Väčšina učebných zdrojov, textov je písaná v prirodzenom jazyku a má neštruktúrovanú formu. Stroje zatiaľ nedokážu úplne pochopiť prirodzený jazyk z dôvodu komplexnosti a variácie jednotlivých jazykov. Spracovanie prirodzeného jazyka je obor, ktorý sa zaoberá spracovaním prirodzeného jazyka strojmi. V našom systéme využívame niekoľko úloh tohto oboru.

Systém sa zameriava na tvorbu poznámok z viet pomocou extrakcie relevantných informácií z nich. Na tvorbu poznámok sa používa hlavne syntaktická analýza viet a extrakcia vzťahov a závislostí medzi slovami viet. Výstupom nami navrhnutého systému sú personalizované poznámky. Systém implementuje prvky interaktivity, ktoré umožňujú používateľovi modifikáciu automaticky vytvorených poznámok. Na základe zmien sa systém ,,naučí'' nove pravidlá ako spracovávať vety. Tieto nové pravidlá zohľadní pri nasledujúcom spracovávaní zdrojov.

\ifthenelse {\boolean{bachelor}}
{
	%\section{Introduction}
	\subsection{Motivácia}
}
{
	%\chapter{Introduction}				
	\chapter{Motivation}
}
Študenti musia často spracovávať veľké množstvá dát, pričom dôležitá je pre nich iba časť dát, ktorá obsahuje relevantné informácie. Proces selekcie relevantných informácií im zaberá pomerne dosť času. S pomocou nami navrhnutého systému by si vedeli z dát vytvoriť poznámky a tak získať podmnožinu, pre nich relevantných informácií, rýchlejšie. Vytvorené poznámku by boli personalizované, teda priamo prispôsobené tvorbe poznámok používateľa a tým väčšmi zredukuje potrebný čas, keďže používateľ nebude musieť získané  relevantné informácie ešte následne upravovať. Získaný čas by mohli využiť efektívnejšie.

%Internet je v dnešných dňoch zaplnený obrovským množstvom dát a informácií. Mnohé z týchto dát sa na internete vyskytujú mnohonásobne, či už v identickej podobe alebo sú podobné. Avšak, čím ďalej tým viac z týchto informácií vyskytujúcich sa na internete, sú informácie irelevantné.
%
%Stáva sa to až príliš často a každý už zažil situáciu, kedy hľadal informácie na konkrétnu tému a musel sa ,,prehrabať'' kopou nepodstatných dát a informácií, ktoré mu boli ponúkané. Stáva sa to medzi všetkými kategóriami používateľov na internete.
%
%Jednou z majoritných kategórií používateľov, ktorí sa s takouto situáciou stretávajú denne sú študenti. Študenti všetkých škôl, od základných až po univerzity, získavajú informácie na učenie, projekty alebo zadania primárne z internetu alebo učebných textov kníh. Keď musia prechádzať obrovské množstvá dát z rôznych zdrojov, je to náročne, často až frustrujúce a berie im to veľa času.
%
%Učebné texty sú prevažne v neštruktúrovanej forme a v prirodzenom jazyku. Pre stroje je mnohokrát náročné správne interpretovať tieto informácie. Jedným z hlavných dôvodov je, že každý jazyk je odlišný a obsahuje špecifické charakteristiky, ktoré môžu byť napríklad slovosled vety, gramatické kategórie slov, ale aj vetné členy a vzťahy medzi nimi.
%
%Tieto, ale aj mnohé iné charakteristiky jazyka sa dajú využiť pri jeho spracovaní a reprezentáciu do podoby, s ktorou vedia aj stroje jednoducho pracovať. Takýto proces sa nazýva \textit{spracovanie prirodzeného jazyka} (angl. Natural Language Processing - NLP). Spracovanie prirodzeného jazyka má viacero aplikácií, z ktorých sú to napríklad preklad jazyka, vytiahnutie najpodstatnejších entít z textu, prípadne aj štatistika ich výskytu a mnohé ďalšie.
%
%My posunieme spracovanie prirodzeného jazyka ešte o kúsok ďalej a budeme sa zaoberať ako pomôcť študentom so spracovaním veľkého množstva informácií, hlavne z učebných textov. Študentom najviac pomôže, ak dokážu rýchlo z textu vytiahnuť tie najpodstatnejšie, najdôležitejšie informácie a údaje, ktoré sa im ďalej budú omnoho ľahšie spracovávať a učiť. Proces určovania a extrakcie najpodstatnejších informácií z učebného textu môžme nazvať spoznámkovávanie.
%
%Zameriame sa hlavne na využitie vetných členov a vzťahov medzi nimi, na určenie najpodstatnejšej a najrelevantnejšej informácie z vety. Takto extrahované informácie následne ponúkneme používateľovi (študentovi).